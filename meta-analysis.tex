\documentclass{article}
\usepackage{blindtext}
\title{Meta-analysis}
\date{8 November 2018}
\author{Max Beerten \and Brent Debleser \and Wouter Devos \and Ben Fidlers \and Simon Gulix \and Tom Kerkhofs}

\usepackage[left=3cm, right=3cm, top=3cm, bottom=3cm]{geometry}
\usepackage{pdfpages}
\usepackage{cite}
\usepackage{eurosym}
\usepackage[utf8]{inputenc}
\usepackage{graphicx}
\usepackage{amssymb}
\usepackage{amsmath}
\usepackage{array}
\usepackage{graphicx}

\usepackage{breakurl}
\usepackage[breaklinks]{hyperref}

\usepackage{apacite}
\usepackage{gensymb}

\usepackage{titlesec}

\setcounter{secnumdepth}{4}

\titleformat{\paragraph}
{\normalfont\normalsize\bfseries}{\theparagraph}{1em}{}
\titlespacing*{\paragraph}
{0pt}{3.25ex plus 1ex minus .2ex}{1.5ex plus .2ex}
\graphicspath{ {./verslag_figuren/} }

\usepackage[numbib]{tocbibind}


\begin{document}

\maketitle

\section{strengths of the report}

Firstly the introduction already had a, maybe not so pronounced, funnel shape. Connections with implementations were made, and it was slowly narrowed down to our own project.

Secondly, the middle part had a clear structure, and the different topics were addressed in a logical order. Figures were used to give a visual representation of the text, and were equipped with a short caption to make clear what was visible. Also, different types of appendices and other literature were carefully referred to according to the regulations.

Lastly, the style used to write the report was impersonal and formal. Informal language and difficult sentence constructions were avoided and differences between American and British English were taken into account. 

\section{weak points of the report}

To begin with, the funnel shape of the introduction, although it was already visible, needs to be even more visible. The example needed to be broaden and there has to be a clear separate section that addresses the structure of the report.

Furthermore, 

\section{improvements}

\section{what have we learned}






\end{document}