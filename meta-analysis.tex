\documentclass{article}

\usepackage[left=3cm, right=3cm, top=3cm, bottom=3cm]{geometry}
\usepackage{pdfpages}
\usepackage{cite}
\usepackage{eurosym}
\usepackage[utf8]{inputenc}
\usepackage{graphicx}
\usepackage{amssymb}
\usepackage{amsmath}
\usepackage{array}
\usepackage{graphicx}

\usepackage{breakurl}
\usepackage[breaklinks]{hyperref}

\usepackage{apacite}
\usepackage{gensymb}

\usepackage{titlesec}

\setcounter{secnumdepth}{4}

\titleformat{\paragraph}
{\normalfont\normalsize\bfseries}{\theparagraph}{1em}{}
\titlespacing*{\paragraph}
{0pt}{3.25ex plus 1ex minus .2ex}{1.5ex plus .2ex}
\graphicspath{ {./verslag_figuren/} }

\usepackage[numbib]{tocbibind}


\begin{document}

\includepdf[pages=-]{Meta_analysis_front.pdf}
\section{Strengths of the report}
A majority of the core elements that are necessary in a good scientific report were already present in the uploaded preliminary report. The most important elements will be discussed in the following section.They will be handled in the order like found in the text.\\
Firstly, the introduction already had a, maybe not so pronounced, funnel shape. It started with the broader perspective and connections with implementations were made. Subsequently, it slowly narrowed down to the true domain of our paper.
Secondly, the middle part existed of a clear structure in which the different topics were addressed in a logical order. Figures were used to give a visual representation of the text, and were equipped with a short caption to make clear what was visible. These pictures really enhance the ability to comprehend the abstract mathematical expressions. Also, different types of appendices and other literature were carefully referred to according to the regulations.
Lastly, the style used to write the report was impersonal and formal. Informal language and difficult sentence structures were avoided and differences between American and British English were taken into account. 

\section{Weak points of the report}

To begin with, the funnel shape of the introduction, although it was already visible, needs to be even more visible. The example needed to be broaden and there has to be a clear separate section that addresses the structure of the report.
\\
Furthermore, 

\section{Improvements}


\section{What have we learned}


Although the seminar was partially based on Dutch reports, and thus the information about sentence constructions, grammatical errors and spelling mistakes wasn't really useful for us. There were still some parts that were valuable for our report. We have learned for example 
that there is a clear difference in the linguistic usage between an essay and a technical scientific report. Furthermore, we got a clear explanation of what is expected to be in the different sections. We got a clear view of what is expected to be in the abstract, introduction and conclusion. Also, the writing tool from the KULeuven is very useful, we will definitely use it during the further development of our report




\end{document}